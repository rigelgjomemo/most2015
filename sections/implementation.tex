% !TEX root =  ../main.tex
\section{Implementation}
\label{sec:implementation}


In this section, we discuss the implementation of the different steps of our approach.

{\color{orange}
\subsection{Implementation Background}
In this subsection, we provide a short background on two techniques and tools that we used in our implementation, namely the IFDS framework and Kaluza. 

\noindent
\textbf{IFDS framework}.
IFDS is a framework for interprocedural data flow analysis that transforms dataflow problems into graph reachability problems \cite{ifds,Bodden:2012:IDA:2259051.2259052}. This framework  is particularly efficient in dealing with interprocedural data flow analysis, and highly customizable to represent different data flow problems. The framework takes care automatically of several general analysis tasks, such as
determination of valid paths on the control flow supergraph (i.e., paths that can potentially be executed at runtime) and of fact propagation. However, to solve a specific analysis problem, it is necessary to formulate it appropriately as an instance of an IFDS problem. In practice, this means defining the analysis information contained in the data-flow \emph{facts} as well as the \emph{rules} that update that information for every node in the control flow supergraph. 

\noindent
\textbf{Kaluza}.
Kaluza is an efficient solver for formulas containing string variables and constraints in the form of string equalities, substring operations, numeric constraints over string lengths, and so on  \cite{kaluza,kaluzaKudzu}. Kaluza natively supports a set of string operations, such as string concatenation, equality, and length equality. \todo{W2.5 Added reference to the Kudzu (Kaluza) paper.} 

} \todo{W1.5. Added implementation background section for IFDS and Kaluza.}

\subsection{Path Computation}
As mentioned previously, the problem of path computation is a typical interprocedural data-flow problem. In our approach, we model this problem as an instance of an interprocedural, finite, distributive, subsets problem (IFDS). This framework is build on top of Soot, a Java optimization framework, due to the many analysis facilities it provides \cite{heros,Bodden:2012:IDA:2259051.2259052,Vallee-Rai:1999:SJB:781995.782008}. In addition, we use the Heros Soot plugin, which provides a fully context-sensitive implementation of the IFDS framework \cite{heros}.
\todo{W2.6}

To prepare the application for use with Soot and HEROS, the Dalvik bytecode is first transformed into Soot's intermediate representation language Jimple. While in theory this transformation may be lossy and not retrieve the original code, these losses are negligible. Jimple is particularly suited to our task, since it provides a three-address and single assignment representation of the code, making it easier to derive the information about paths and perform symbolic execution. In addition, the Soot framework provides many ready-to-use capabilities for code analysis.

In our implementation, we model the preliminary taint propagation step as a data-flow problem as well, and incorporate it into the IFDS instance problem of path computation. This removes the need for running this step separately and improves the efficiency of our implementation. Before proceeding further, we provide a short review of the IFDS framework.

% IFDS relies on a control flow supergraph similar described in Section \ref{sec:approach}. 
% After a user defines the analysis information contained inside a fact and the update rules for different types of nodes, the IFDS framework then proceeds to traverse the supergraph and use those rules to update facts with new information. The graph traversal stops when a fixpoint is reached and no new information can be added to any facts.


\noindent
\textbf{Source statements detection}.
The first step in defining the problem as an IFDS instance is the specification of the \textit{source statements}, which constitute the IFDS analysis entry points. To detect these entry points, a full scan of the Jimple representation of the program is performed and the instructions that perform Intents and Bundle payload extraction (e.g. $getStringExtra$) are identified. Since the set of API calls that Android provides to extract Intent payloads is limited, we use an exhaustive list of method signatures for this task.
%What does the sentence: "A step further...." mean?
A step further is made in order to reconcile extractions of different payload pieces conceptually belonging
 to the same Intent message. 
 After this detection, the program variables whose value is defined in the entry points serve as the initial taint variables. These are indeed the variables appearing in the state $V_I$, whose value can be set by an attacker.
%In most of the cases, this extraction is performed in the $onCreate$ or in the $onBind$ methods, which are the set-up methods for both Android Activities and Services.
% Practically, this means that we will have a direct connection between an attack stimulation (sending an intent message) and the vulnerable code fragments that use them.


%We are not using these entry points later. REMOVE
% To represent an entry point, we use a data structure populated with information about the Android component in which the extractions
% were encountered, the method in which the single piece was first assigned complete with the exact location
% in the method body. An entry point for the payload piece $host$ in the example proposed in Figure~\ref{lst:example} is shown below:
% \begin{verbatim}
%   EntryPoint =
%    {
%      name: "host",
%      component: SendActivity,
%      method: SendActivity$onCreate(),
%      location: line 3,
%      statement:
%       String host = intent.getStringExtra("host");
%    }
% \end{verbatim}
% In this case, the following steps of the analysis, consider the variable $host$ as the tainted variable
% from the \textit{source} statement. %The path identification and computation steps are aimed to find a path between
%variable $url$ and a vulnerable sink, in addition to a value to this variable that makes
%such path traversable.

\noindent
\textbf{Path Computation and Taint Propagation}.
The next step in modeling our problem as an IFDS instance, is to define the information contained inside the dataflow facts and how this information is updated for the different nodes of the supergraph. In our implementation, we use a special definition of a \emph{fact}, which contains both taint propagation and path information. Thus, a \emph{fact} is defined as tuple $F = (input\_vars, tainted\_vars, statements, cond\_statments)$, where $input\_vars$ represents variables from the state $V_I$ whose values is defined in the source statements (namely predefined Android API statements that extract values from intents), $tainted\_vars$ is a set of variables representing the tainted variables, $statements$, is the list of statements from the source statement to the current point in the program, and $cond\_statements$, is the list of all the conditional statements from the source statement to the current point in the program. 

{\color{orange}
Thus, for every program point, the fact associated with that point contains the list of input variables, the list of tainted variables visible in that point, as well as a list of statements contained in the paths from the source statements to that program point. The taint is therefore propagated to a variable by adding that variable to the list.
} \todo{W2.3a}
%Do we need this? We do not use it later in the discussion
  %In addition, each fact stores the name of the component containing the source statement as well as the name, the type and the method signature containing the source statement. 
% For instance, when the analysis reaches line 3 in Figure \ref{lst:example} where the $host$ variable is extracted from the intent, a new fact $Fact$ is created:

% \begin{verbatim}
% Fact = { input-variable: "host",
%   tainted-variables: "host"
%   statements: host=
%         intent.getStringExtra("hostname")
%   conditional-statements: <empty>
%   context: line 3 }
% \end{verbatim}

During analysis, the IFDS framework takes care of traversing the supergraph and update the facts associated with each node using user-defined \emph{rules}. These rules are different for every node and described below.
\begin{itemize}
 \item \textit{Normal Rules:} These functions define how fact information is updated for nodes different from method calls. In this case, we add a statement to the $statements$ list if: either the $input\_vars$ or one of the $tainted$ variables is used by the statement. A new variable is added to the $tainted$ set if its value is obtained by using one of the $input\_vars$ or a tainted variable. \todo{W2.3b.}
 \item \textit{Call Rules:} These rules define how fact information is updated for procedure calls. In this case, the call statement is added to the $statements$ list, if the $input\_var$ or one of the tainted variables are contained in the list of arguments. In addition, this rule is used to add to the set of tainted variables the formal parameters of the callee that correspond to tainted variables in the call.
 \item \textit{Return Rules:} The purpose of a return rule is to propagate the information discovered inside the body of a called method to the caller. Using this rule, the taint information is therefore propagated to the variables at the caller site. For instance, since the return value of the method $toBase64$ is tainted, the variable $b64File$ is added to the list of tainted variables.
\end{itemize}
% Both normal and call rules serve also to create a new fact, when a method that extracts data from an intent is encountered.

% % \todo{Added}
% % For instance, the fact $F$ is created when the analysis reaches line 4. The statement at line 5 where the $filename$ variable is used, is also added to the $tainted-statements$ set. In addition, the $b64File$ name is added to the set of $tainted$ in $F$, since the statement contains the $filename$ on the right hand of the assignment.
% % The statement at line 7 is added as well to the $tainted-statements$, since the alias $b64File$ is matched. The variable $httpPar$ is also added to the $tainted$ set, since $b64File$ now appears on the right hand side of the statement.

% After the definition of these rules is provided, the IFDS framework traverses the super-graph and executed them depending on the type of statements encountered. For instance, if the statements contains a procedure call, call rules are executed. We added an exception to this behavior in our implementation: When the statement contains a library method call, we execute normal rules rather than call rules to prevent the IFDS framework from traversing the body of those calls.

\subsection{Symbolic Execution Implementation}
Given the paths identified inside the facts by the previous step, we match each statement to one of the productions of the symbolic formula grammar described in Section \ref{sec:approach} and add it as a term to the rest of the formula. In particular, the rules for each of the productions are described below. {\color{orange} In this step, we also perform variable name rewriting, to $flatten$ the objects and to extract constraints among $String$ variables. This renaming is performed by prepending to the name of a variable, the name of the method it is declared in and the name of the corresponding object (if available). For instance, the variable $p$ inside the function $toBase64$ is renamed to $toBase64\_p$.} \todo{W2.2b}
\begin{itemize}
\item \textit{Assignment statements:} Every assignment statement in the path is transformed into an equality constraint.
\item \textit{Branching statements:} For every branching statement, symbolic execution is split into two paths and the condition of the branching statement is added to the true branch, while its negation is added to the false branch.
For instance, when the $if$ statement in line 28 of Fig. \ref{lst:example} is encountered, the condition $toBase64\_p==null \| toBase64\_p==``''$ is added to the formula representing the path that contains the $then$ part of the statement, while the condition $!(toBase64\_p==null \| toBase64\_p==``'')$ is added to the path that contains the $else$ part of the statement.
\item \textit{User-defined procedure calls:} When a call to a user defined method is encountered, we first rename the local variables of the method by prepending the name of the function to avoid duplicates, then add equality constraints between arguments and formal parameters. Next, we proceed to symbolically execute the function. For instance, since the variable $file$ is passed as an argument to $toBase64$, we add the constraint $``file == toBase64\_p''$ to the formula. If the function returns a value that is assigned to a variable, we add an equality condition between them as well. For instance, the condition added after the return statement of $toBase64$ is $b64File==toBase64_b$.
\item \textit{Library method calls:} Unsolvable library method calls are replaced by a special term whose purpose is to not introduce any constraints to its arguments. 
\end{itemize}

% !TEX root =  ../main.tex

\subsection{Exploit Proof Generation}
 \label{section:kaluzaTranslation}

As mentioned previously, to generate an exploit as a malicious state input state $V_I$, we chose to use the Kaluza constraint solver. Kaluza natively supports several string operations. For other operations, not natively supported by Kaluza, a translation system for a set of Java (Jimple) standard library methods was built, which focuses on string and integer constrains. This set together with the set of operations natively supported implements the \emph{solvable} library methods previously discussed. Some examples of these custom translations are listed in Table~\ref{table:tabkaluza}, using regular definitions. 

% \begin{table}[t]
% \small
%    \centering
%     \begin{tabular}{|l|l|}
%       \hline
%       \textbf{Java} & \textbf{Kaluza formulation}\\
%       \hline
%       $a.contains("test")$ & $a == /.*test.*/$ \\
%       % \hline
%       % $a.indexOf("test")$ & $a \rightarrow T1 T2 T3$ \\
%       %             & $T1 \nrightarrow test$ \\
%       %             & $T2 \rightarrow test$ \\
%       %             & $T3 \rightarrow *$ \\
%       %             & $a\_indexOf \rightarrow Len(T1)$ \\
%       \hline
%       $new\_a = a.replace("test",$ & $a == T1 T2 T3$ \\
%        $"newTest")$ & $T2 \rightarrow test$ \\
%                       & $T4 \rightarrow newTest$ \\
%                       & $new\_a \rightarrow T1 T4 T3 $ \\
%                       & $T1 \rightarrow *$ \\
%                       & $T3 \rightarrow *$ \\
%       \hline
%     \end{tabular}
%     \caption{Kaluza constraints formulation example}
%     \label{table:tabkaluza}
%   \end{table}



\begin{table}[t]
   \centering
    \begin{tabular}{|l|l|}
      \hline
      \textbf{Java} & \textbf{Kaluza formulation}\\
      \hline
      $a.contains("test")$ & $a \in CB(/.*test.*/, 0);$ \\
      \hline
      $a.indexOf("test")$ & $a := T1 . T2;$ \\
                  & $0x0 == Len(T1);$ \\
                  & $T2 := T3 . T4;$ \\
                  & $T3 := T5 . T6;$ \\
                  & $T6 == "test";$ \\
                  & $T5 \notin CB(/test/, 0);$ \\
              & $a\_indexOf == Len(T5);$ \\
      \hline
      $new\_a = a.replace("test",$ & $a := T1 . T2;$ \\
       $"newTest")$ &  \\
                      & $T2 := T3 . T4;$ \\
                      & $T3 := T5 . T6;$ \\
                      & $T5 \in CB(/test/, 0);$ \\
                      & $T7 \in CB(/newTest/, 0);$ \\
                      & $T8 :=  T9 . T4;$ \\
                      & $T9 :=  T7 . T6;$ \\
                      & $new\_a := T1 . T8;$ \\
      \hline
    \end{tabular}
    \caption{Kaluza constraints formulation example (CB = CapturedBrack)}
    \label{table:tabkaluza}
  \end{table}

\todo{W2.4a. Reinstated Daniele's original table.}.
  %These translations rely on Kaluza regular expressions and constraints. For instance, the $String.replace$  method is translated by producing a conjunction of constraints:
  %the string ($var_a$) is divided into parts ($T1$ and $T2$ in the example) in order to identify the correct
  %string part to substitute ($T5$). The new string is then defined to contain the replaced part
  % ($T7$) in addition to the other non-matching string parts that do not need replacement.
  % A similar approach is adopted for the $indexOf$ method: instead of replacing the string, the formulation identifies the correct position of the first occurrence of the lookup string in the original string by identifying the matching string part.
  % For sake of clarity we present a line-by-line discussion of the $String.replace$ formulation.
  % Line $1$ defines the variable $\$var\_a$ as the concatenation of two substrings ($T1$ and $T2$).
  % Since the string to replace be found in the middle of the string, this first
  % splitting is not exhaustive. At Line $2$ and $3$ we proceed further splitting the substring $T2$ in three
  % pieces: $T5$, $T6$ and $T4$. Thanks to this string representation and not bounding the lengths of the pieces
  % we can handle any positioning of the replacement string.
  % At line $4$ we use $CapturedBrack$ (a regular expression helper) to explicitly declare which one is the part
  % containing the string to be replaced. Similarly at line $5$ we ensure to have a new string piece ($T7$)
  % including the replacement string.
  % The remaining constraints (Lines $6-8$) compose the new string by concatenating the string pieces,
  % including $T7$.


  % Since Java and Kaluza are both statically typed, we did not had to care about type casting
  % explicitly. For this reason the lookup of the method intended to be translated is straight forward
  % and can be done by exactly matching the method signature. The Kaluza transformation is consistent
  % in different invocation of the same  method signature.

  % It is important to note that, in general, not all of the methods included in symbolic formulas produced by the paths computation step can be translated into Kaluza. Examples include API calls with types that cannot be cast to a Kaluza type. For instance, in our example code in Figure~\ref{lst:example}, this happens with the instruction Base64Encoder.toString(bytes). Another typical example are API calls that we consider as atomic instructions.


  For library methods that can not be translated directly in Kaluza (e.g., Base64Encoder.toString(bytes)), a report is created in output, and either a Kaluza approximation of their functionality is manually built or they are represented by a special term that places no constraints on the values of their variables. An example of such approximation is the $split$ method, which is a utility function to divide a string into pieces separated by a substring given in input to the function. $Split$ returns an array, and arrays are hard to represent in Kaluza because they are defined as an unknown number of variables, while Kaluza accepts only defined numbers of variables. Our approximation consists in producing the entire string instead of an array of parts as returned value of the method.

% \begin{itemize}
%   \item \textbf{Alias binding}: by binding parameters to arguments we obtained a
% linear and simple representation of the intra-procedural flow. As we can see at Line~$1$ in the wxample, we adopted the
% method signature as namespace to local variables so to have a unique value name
% for each variable in the code flow. In order to produce this bindings we use the $<tainted>$ data structure contained in the fact.
%   \item \textbf{Constraints translation}: we implemented a class of utility
% functions to translate string comparisons methods to a formal Kaluza representation. We directly extract the comparison information from the fact's $<conditional-statemets>$ data structure. In order to correctly detect the control flow edge leading to the vulnerable statement (vulnerable sink), we locally explore the CFG (Control Flow Graph) and we possibly flip the constraint condition (Line~$2$ of the example).
% \end{itemize}

% \begin{itemize}
%   \item \textbf{Graph building}: as first step of the process, we are interested in producing an
%   unified Graph of the original control flow graph by removing all the statements that does not
%   involve any of the statements that operates or include either the \emph{base variable} or one
%   of its aliases. The resulting sub-graph is then obtained by recursively merging all the methods' graphs
%   starting from the sink to the \emph{base variable}.

%   Here below is presented a sketch of the algorithm used to obtain the desired graph.
%   Where:
%   \begin{itemize}
%   \item $graph$ is the resulting graph variable
%   \item $sink$ is the vulnerable statement
%   \item $statementsSet$ is the list of tracked statements from the IFDS analysis
%   \end{itemize}
%   The algorithms works as follows: starting from the vulnerable statement point in the code
%   the control flow graph is traversed backward (w.r.t. normal code execution order) jumping
%   from callee to callers until the \emph{entry point} variable is reached.

%   $extractMethod$ is the algorithm wrapper function taking care of managing inter-procedural calls.
%   After adding the vulnerable statement the algorithms iterates over the callees until the newly generated
%   graph has not been changed (convergence point).

%   \lstset{numbers=left, basicstyle=\ttfamily\footnotesize, caption=Subraph building., label=codelisting}
%   \begin{lstlisting}
% graph;
% sink;
% statementsSet; // list of captured statements
% extractGraph() {
%  context; // sink's method
%  contextGraph;

%  graph.addNode(sink);

%  addPredecessors(contextGraph, sink, sink);

%  repeat {
%   headMethod= methodOf(lastUpdatedHead);
%   caller= callerOf(headMethod);
%   context= methodOf(callerStmt);
%   contextGraph= extractGraph(context);

%   if (caller in statementsSet) {
%    addPredecessors(contextGraph, callerStmt, prevHead);
%   }

%  } until (graph.getHead() != lastUpdatedHead);
% }

% addPredecessors(contextGraph, currentStatement, graphHead) {
%  for (predecessor, i :
%     currentStatement.getPredecessors()) {
%    if (!currentStatement in statementsSet) {
%     addPredecessors(contextGraph, predecessor, graphHead);
%     return;
%    }
%    if (current statemet branches) {
%     if (i == 1) {
%      addNodesAndEdge(graphHead,
%         <Empty>);
%     }

%     addNodesAndEdge(predecessor, graphHead);

%     if (i == 0) {
%       addNodesAndEdge(graphHead,
%         <Empty>);
%     }
%    } else {
%      addNodesAndEdge(predecessor, graphHead);
%    }

%    addPredecessors(contextGraph, predecessor, newGraphHead);
%  }
% }

%   \end{lstlisting}
%    $addPredecessors$ iterates over statements inside a specific method body and it is in charge
%   of reconstructing intra-procedural code points from the original method's body to newly constructed graph.
%   This method is invocated from the precise program point from which the previous method was invocated
%   until the first statement in the method body.

%   The function first checks whether the current statement was tracked in the previous analysis step or not,
%   i.e. such statement affects somehow the \emph{base variable} (or one of its aliases) or not.
%   If a match is encountered and the statements is not a merging point (the graph branches),
%   the current statements is simply added to the graph and the next statement is taken in consideration.
%   If the statement is a merging point, since the then-else branches are positional in the graph semantic,
%   the right position for the next statement has to be preserved. This is obtained by checking
%   the index of the current statement in the predecessors list (0 or 1).

%   It has to be noticed that:
%   \begin{itemize}
%   \item the first statement in a method body has no predecessors
%   \item points in the code in which if-then-else blocks merges have two predecessors
%   \item all other statements have only one predecessor
%   \end{itemize}

%   $addNodesAndEdges$ method (not presented here) is a simple utility method to add the edge
%   between the two nodes in the right position.


  % In order to avoid clashes, local variable names are transformed by prefixing to them their method name.
  % For example variable $v$ of method $doSomethingElse$ presented in the example \ref{sec:graph_buinding} is transformed to
  % the variable name $doSomethingElse_v$.

After the constraint solver processes the translated formula, it provides a set of solutions for the ranges of variables in $V_I$ for which the formula $F_p \wedge V_E$ is satisfiable, {\color{blue} where $V_E$ is also expressed using the Kaluza language, which provides the opportunity to define several patterns for the values of the parameters at the sinks.} Conversely, an unsatisfiable result produced by the solver means that the vulnerable sink cannot be reached, in practice, at run-time. For instance, the following example shows a Kaluza formula derived from one of the studied applications. In this example, after ensuring that the input value $source$ length is greater than 0, the prefix is stripped out (if any) in $tmp$. This $tmp$ variable is then matched against a regular expression for validation purposes. Our tool was able to traverse the control flow graph with the Jimple representation of this code fragment, and in the end obtain the solution $444494 4944$.
\lstset{numbers=left,xleftmargin=1cm, basicstyle=\ttfamily\scriptsize, breaklines=true}
\begin{lstlisting}
$source.length > 0
$IF(source.startsWith("+1")) { $tmp := $source.substring(2, $source.length - 1) }
IF((not $source.startsWith("+1"))) { $tmp := $source }
$tmp.match("/([2-9][0-8][0-9])\ *([2-9][0-9][0-9])\ *([0-9][0-9][0-9][0-9])/")
\end{lstlisting}

\todo{W2.4b}


\subsection{Approximations and limitations}

For simplicity, we used several approximations in our approach. We discuss their impact in the following paragraphs.

\emph{Untainted input variables}. Motivated by efficiency considerations, we chose to ignore variables whose values cannot be affected by the variables input via intents. While allowing us to prune a portion of the control flow super-graph by removing statements that do not use tainted variables, this choice may also reduce the precision of our approach. In fact, the value of the outputs at the sink may also depend on these variables. {\color{orange} The overall result of this choice is that the untainted variables may appear in constraints containing tainted variables. For instance, if a statement $x=s+t;$ appears in the code, with $s$ untainted and $t$ tainted, the constraint $x==s+t$ will be added by the symbolic execution to the symbolic formula. However, since the solver has no constraints related to $s$, it assumes that $s$ can have any possible value.} \todo{W2.2a}

This approximation may lead to \textit{false positives} where an input state $V_I$ is computed statically starting from an exploit state $V_E$, while at run time, the presence of these untainted variables in the computation path may induce a state different from $V_E$ at the \textit{sink} statement.

{\color{orange}
\emph{Attack Effectiveness}. The effectiveness of this attack depends partly on the Android communication system, on the installed apps in the device, as well as on user attention. In particular, when an intent is sent, if several activities have registered to as receivers for that intent type, Android will present a list of choices to the user. Thus, if malicious intents are caught in this way, the attack may fail. However, we note that this type of behavior may be bypassed by an attacker by sending explicit intents, where the receiving component is explicitly named.}
\todo{W1.4. We believe that the reviewer's observation is correct, and that the way in which activities are started may be a limitation and needs to be mentioned. However, the presentation of a list of apps to choose from is not very common. In addition, an attacker may bypass the list by creating an explicit intent where the receiving component is explicitly mentioned.}

% \emph{Loop unrolling and library method calls}. Loops and libraries are common problems that affect the efficiency of symbolic execution. By unrolling loops at most one time and by choosing to model API procedure calls as atomic instructions, one of the risks is missing additional computations (that is constraints) on the tainted variables. This again may lead to \textit{false positives}, where given a desired malicious output $O_m$, we obtain a too large malicious input set $I_a$ because of the fewer constraints.

% !TEX root =  ../main.tex
\subsection{Attack app construction}
Having generated the exploit automatically, we take a step further, and 
generate an attack application that exploits the vulnerable application. 
An effective attack is automatically prepared in the form of a well-crafted application able to send
Intent messages to the right target, at the right moment in time and carrying the malicious payload.
It is entirely possible to present such attack application as a simple utility application (such as a torch), 
which runs a malicious \emph{service} behind the scene.

Our malicious service embeds the exploit strings obtained from the previous steps in a list of pre-populated
Intent messages ready to be sent, one for each demonstrated vulnerability.
In order for the attack to be successful, we enhance the service logic and domain knowledge to
obtain the best attack scenario possible: the user perception of the environment during the attack must
be as transparent as possible.

For each of the messages, the service first needs to check whether the target vulnerable application
is installed on the system or not. We do this through the Android \emph{Package Manager API} offering
the $getInstalledApplications$ method. The invocation with the $PackageManager.GET\_META\_DATA$ specified as argument
returns a complete list of all the applications installed on the phone. The list contains several meta-data such as
the application's package name or the application's launch Intent message.

In the next step of the attack the application checks if the application is currently active (in foreground) on the device.
This is particularly important for two reasons: first we limit the context switching consequences of presenting
to the user a completed unrelated content (e.g. a different application screen), then we can assume
that the user uses the vulnerable application and hence that the user is logged in at the moment of the attack.
Also in this case, Androids offers an API: the \emph{Activity Manager API},
in particular the $getRunningAppProcesses$ method that returns a list of $RunningAppProcessInfo$ containing the foreground/background information along with the packages Application process package information, that are used to identify the application. When it ensures that a victim application is installed and running, the attacker app easily sends the malicious Intent message to trigger the attack.

% By separating these two checks we can leave our application silent, avoiding to poll the running processes, if none of the vulnerable applications is currently installed on the device. Indeed, since application installations are not so common, the installation check can be performed in reasonably long time intervals.

